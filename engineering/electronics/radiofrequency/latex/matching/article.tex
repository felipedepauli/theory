\documentclass[12pt,a4paper]{article}

%   Configura as margens que nem a bunda dele.
    \usepackage[left=1.5cm, right=1.5cm, top=1.5cm, bottom=1.8cm]{geometry}

%   Traduz as frases autogeradas para português. Também faz a separação silábica.
    \usepackage[portuguese]{babel}

%   Configura para que todas as seções iniciem com um parágrafo identado.
    \usepackage{indentfirst}

%   Facilita os processos de alinhamento. São eles:
    \usepackage{titlesec}
    \usepackage{float}
    %\usepackage{setspace}

%   Permite a utilização de gráficos (inserir imagens)
    \usepackage{graphicx}
    \graphicspath{{./images/}}

%   Várias fórmulas matemáticas
    \usepackage{amsmath}

    \raggedbottom
    \usepackage[bottom]{footmisc}
    \setlength{\parindent}{1.2cm}
    \linespread{1}
    \usepackage[font=footnotesize,labelfont=bf]{caption}

    % Aqui o bicho pega. Os circuitos bonitos são gerados por ele
    \usepackage{tikz, tkz-euclide}
    \usepackage{circuitikz}
    \usepackage{siunitx}

%   Configurações das colunas
    \usepackage{multicol}
    \setlength{\columnsep}{30pt}

%   ## CHANGING FONTS
    \renewcommand*\ttdefault{txtt}
    \renewcommand*\familydefault{\ttdefault}
    \usepackage[T1]{fontenc}
    \usepackage{everysel}
    \EverySelectfont{%
    \fontdimen2\font=0.4em% interword space
    \fontdimen3\font=0.2em% interword stretch
    \fontdimen4\font=0.1em% interword shrink
    \fontdimen7\font=0.1em% extra space
    \hyphenchar\font=`\-% to allow hyphenation
    }

% Vai começar a palhaçada
\begin{document}
\begin{center}
\Large Segurança - VPN
    \vskip .3cm
     \large Felipe Camargo de Pauli
    \end{center}
    \vskip .3cm
\begin{multicols*}{2}
\section*{\center{Resumo}}
\normalsize
\titlespacing\section{0pt}{3mm}{1mm}[0mm]
\setlength{\parskip}{0cm}
Abstract
\setlength{\parskip}{.2cm}
% ##############################################
% ############ INTRODUÇÃO ######################
\section{Introdução}
    {

   Como nós bem sabemos de toda esta pataquada 




    }
    {
    \unskip\parfillskip 0pt
    Também percebeu-se que a cada 5 múltiplos de RC segundos, tem-se ou o capacitor carregado ou descarregado. Esse período tem uma denominação própria: constante de tempo $\tau$.\cite{sadiku}\par}

\section{Métodos}

    Um circuito contendo um resistor, um capacitor e um gerador de tensão foi utilizado nesta experiência. Também foram utilizados um aparelho para medir as resistências dos componentes e capacitância do capacitor, um multímetro digital para medir a tensão ao longo do tempo, e um celular para gravar cada curva de carregamento e descarregamento.

    Inicialmente montou-se o circuito de acordo com o diagrama abaixo.
    
    % Aqui vai rolar o Circuitkx
        \begin{figure}[H]
            \centering
            \begin{tikzpicture}[american voltages]
                \tkzDefPoints{
                    0/0/A,
                    0/4/B,
                    3/4/C,
                    3/3.5/D,
                    3/2/E,
                    3/0.5/F,
                    3/0/G}
                \draw (B) to[V, v_=20$V{pp}$] (A);
                \draw (B) to (C);
                \draw (C) to (D);
                \draw (D) to[C=100uF] (E);
                \draw (E) to[R=100k\si{\ohm}] (F);
                \draw (F) to (G);
                \draw (G) to (A);
                \draw[thin, <-, >=triangle 45,path picture={
                    \node[anchor=center]  at (path picture bounding box.center) {$i_c$};
            }] (2,1.3) arc (-60:170:0.8);       
            \end{tikzpicture}
            \caption{O circuito RC é ligado a uma fonte de tensão CC que provoca o carregamento do capacitor. A corrente, que começa em intensidade máxima, diminui exponencialmente com o tempo à medida que o capacitor é carregado.}
        \end{figure}
    
    Com o circuito montado, ligou-se o gerador de tensão, o qual foi configurado para fornecer 20V para o circuito. Este valor foi definido pois tratava-se do valor de tensão nominal do capacitor. Este circuito foi mantido aberto até o início da primeira tomada de medidas.
    
    O multímetro foi colocado em paralelo ao capacitor e o celular posicionado com a câmera ligada filmando o visor deste multímetro. O circuito foi fechado e o capacitor começou a carregar. Após um minuto, o circuito foi desligado.
    
    Para otimizar o tempo, aguardou-se o capacitor se estabilizar em 19V, quando o circuito era fechado novamente, mas agora sem a fonte de tensão. Ou seja, o circuito tinha o capacitor fornecendo tensão para o resistor.
    \begin{figure}[H]
        \centering
        \begin{tikzpicture}[american voltages]
            \tkzDefPoints{
                0/0/A,
                0/4/B,
                3/4/C,
                3/3.5/D,
                3/2/E,
                3/0.5/F,
                3/0/G}
            \draw (B) to(A);
            \draw (B) to (C);
            \draw (C) to (D);
            \draw (D) to[C=100uF] (E);
            \draw (E) to[R=100k\si{\ohm}] (F);
            \draw (F) to (G);
            \draw (G) to (A);
            % \draw[-latex] (-.5,2.5) -- (-.5,1.5);
            \draw[thin, <-, >=triangle 45,path picture={
            \node[anchor=center]  at (path picture bounding box.center) {$i_d$};
            }] (0.8,2.4) arc (160:-70:.8);
        \end{tikzpicture}
        \caption{O circuito RC é desconectado da fonte e, a partir de então, passa a ser o fornecedor de energia para o circuito, gerando uma corrente de sentido inverso a propelida pela fonte anteriormente.}
        \label{circuitoDescarga}
    \end{figure}
    
    Diferente do circuito anterior, a corrente fluiu em sentido contrário até o momento de total descarga do capacitor.


\section{Resultados e Discussão}
    
    Com o múltímetro ligado em paralelo ao capacitor, as medidas de tensão foram tomadas a cada cinco segundos e inseridas na seguinte tabela. Os valores teóricos foram calculados por meio das equações \eqref{eq:coisa1} e \eqref{eq:coisa2}, enquanto o erro relativo foi calculado por meio da relação a seguir:

    \begin{equation}
        Err_{relativo} = \frac{|V_m-V_t|}{V_t}
        \label{eq:coisa1}
    \end{equation}

    A incerteza foi calculada utilizando a equação de derivadas parciais.

    \begin{equation}
        \sigma = \overline V \left(\frac{\partial \overline V}{V_1}\right)^2\sigma_1^2+...+\left(\frac{\partial \overline V}{V_5}\right)^2\sigma_5^2
        \label{eq:coisa2}
    \end{equation}

    
    
\begin{thebibliography}{99}
    \bibitem    {mit}
                \textbf{AGARWAL, A. \& LANG, G. H.}
                Foundations of Analog and Digital Electronic. Circuits. San Francisco, Elsevier Inc, 2005.
    \bibitem    {sadiku}
                \textbf{ALEXANDER, C. K. \& SADIKU, M. N. O.}
                Fundamentos de Circuitos Elétricos. Porto Alegre, Bookman Cia. Editora, 2003.
    \bibitem    {halliday}
                \textbf{HALLIDAY, D. RESNICK, R.}
                Fundamentos de Física: Eletromagnetismo. 9a Ed. Rio de Janeiro: Livros Técnicos e Científicos, 2012. Vol 3.
    \bibitem    {edo}
                \textbf{ZILL, D. G.}
                A First Course in Differential Equations, with Modeling Applications. 9a Ed. California, Brooks/Cole.


\end{thebibliography}
\end{multicols*}
\end{document}